%%%%%%%%%%%%%%%%%%%%%%%%%%%%%%%%%%%%%%%%%
% Structured General Purpose Assignment
% LaTeX Template
%
% This template has been downloaded from:
% http://www.latextemplates.com
%
% Original author:
% Ted Pavlic (http://www.tedpavlic.com)
%
% Note:
% The \lipsum[#] commands throughout this template generate dummy text
% to fill the template out. These commands should all be removed when 
% writing assignment content.
%
%%%%%%%%%%%%%%%%%%%%%%%%%%%%%%%%%%%%%%%%%

%----------------------------------------------------------------------------------------
%	PACKAGES AND OTHER DOCUMENT CONFIGURATIONS
%----------------------------------------------------------------------------------------

\documentclass{article}

\usepackage{fancyhdr} % Required for custom headers
\usepackage{lastpage} % Required to determine the last page for the footer
\usepackage{extramarks} % Required for headers and footers
\usepackage{graphicx} % Required to insert images
\usepackage{lipsum} % Used for inserting dummy 'Lorem ipsum' text into the template
\usepackage{mathtools, bm}
\usepackage{amssymb, bm}


% Margins
\topmargin=-0.45in
\evensidemargin=0in
\oddsidemargin=0in
\textwidth=6.5in
\textheight=9.0in
\headsep=0.25in 

\linespread{1.1} % Line spacing

% Set up the header and footer
\pagestyle{fancy}
\lhead{\hmwkAuthorName} % Top left header
\chead{\hmwkClass\ (\hmwkClassInstructor\ \hmwkClassTime): \hmwkTitle} % Top center header
\rhead{\firstxmark} % Top right header
\lfoot{\lastxmark} % Bottom left footer
\cfoot{} % Bottom center footer
\rfoot{Page\ \thepage\ of\ \pageref{LastPage}} % Bottom right footer
\renewcommand\headrulewidth{0.4pt} % Size of the header rule
\renewcommand\footrulewidth{0.4pt} % Size of the footer rule

\setlength\parindent{0pt} % Removes all indentation from paragraphs

%----------------------------------------------------------------------------------------
%	DOCUMENT STRUCTURE COMMANDS
%	Skip this unless you know what you're doing
%----------------------------------------------------------------------------------------

% Header and footer for when a page split occurs within a problem environment


% Header and footer for when a page split occurs between problem environments


\setcounter{secnumdepth}{0} % Removes default section numbers
\newcounter{homeworkProblemCounter} % Creates a counter to keep track of the number of problems

\newcommand{\homeworkProblemName}{}
\newenvironment{homeworkProblem}[1][Problem \arabic{homeworkProblemCounter}]{ % Makes a new environment called homeworkProblem which takes 1 argument (custom name) but the default is "Problem #"
\stepcounter{homeworkProblemCounter} % Increase counter for number of problems
\renewcommand{\homeworkProblemName}{#1} % Assign \homeworkProblemName the name of the problem
\section{\homeworkProblemName} % Make a section in the document with the custom problem count
\enterProblemHeader{\homeworkProblemName} % Header and footer within the environment
}{
\exitProblemHeader{\homeworkProblemName} % Header and footer after the environment
}

\newcommand{\problemAnswer}[1]{ % Defines the problem answer command with the content as the only argument
\noindent\framebox[\columnwidth][c]{\begin{minipage}{0.98\columnwidth}#1\end{minipage}} % Makes the box around the problem answer and puts the content inside
}

\newcommand{\homeworkSectionName}{}
\newenvironment{homeworkSection}[1]{ % New environment for sections within homework problems, takes 1 argument - the name of the section
\renewcommand{\homeworkSectionName}{#1} % Assign \homeworkSectionName to the name of the section from the environment argument
\subsection{\homeworkSectionName} % Make a subsection with the custom name of the subsection
\enterProblemHeader{\homeworkProblemName\ [\homeworkSectionName]} % Header and footer within the environment
}{
\enterProblemHeader{\homeworkProblemName} % Header and footer after the environment
}
   
%----------------------------------------------------------------------------------------
%	NAME AND CLASS SECTION
%----------------------------------------------------------------------------------------

\newcommand{\hmwkTitle}{Bachelor Project\ \#1} % Assignment title
\newcommand{\hmwkDueDate}{Monday,\ January\ 1,\ 2012} % Due date
\newcommand{\hmwkClass}{IN\ 101} % Course/class
\newcommand{\hmwkClassTime}{10:30am} % Class/lecture time
\newcommand{\hmwkClassInstructor}{Pr. Serge Vaudenay} % Teacher/lecturer
\newcommand{\hmwkAuthorName}{Max Premi} % Your name

%----------------------------------------------------------------------------------------
%	TITLE PAGE
%----------------------------------------------------------------------------------------

\title{
\vspace{2in}
\textmd{\textbf{\hmwkClass:\ \hmwkTitle}}\\
\normalsize\vspace{0.1in}\small{Due\ on\ \hmwkDueDate}\\
\vspace{0.1in}\large{\textit{\hmwkClassInstructor\ \hmwkClassTime}}
\vspace{3in}
}

\author{\textbf{\hmwkAuthorName}}
\date{} % Insert date here if you want it to appear below your name

%----------------------------------------------------------------------------------------

\begin{document}

\maketitle

%----------------------------------------------------------------------------------------
%	TABLE OF CONTENTS
%----------------------------------------------------------------------------------------

%\setcounter{tocdepth}{1} % Uncomment this line if you don't want subsections listed in the ToC

\newpage
\tableofcontents
\newpage

%----------------------------------------------------------------------------------------
%	PROBLEM 1
%----------------------------------------------------------------------------------------

% To have just one problem per page, simply put a \clearpage after each problem

\section{Introduction}
	The motivation of this project is first and foremost to improve the Linear attack on the Hill cipher , by changing the recovery of the key modulo 26 and then see the possible algorithm to improve the FFT.\\
Let's briefly recall how this attack works.\\
You get the plaintext into modulo 2 , then with the aid of vectors , and bias \\
bias(X)= $\varphi_{X}(\frac{2\pi}{p})$ in $\mathbb{Z}/26\mathbb{Z}$\\
With the fact that the plain text is of length d , we can write $\lambda$ X that represent the plain text , where X is a vector column.\\
Then thanks to the bias , we found correspondence between $\lambda$ and $\mu$ (which is the same vector but for the cipher text). Wa actually get $ \mu = (K^T)^{-1} \times\lambda $\\
Then with this formula and the approximation of all the vector $\mu$ , we get the column of the key matrix.\\
Then we need to find the correct order in the key matrix , we use algorithm 1\\


\section{Key recover modulo 26}
So now that we have the key matrix in $\mathbb{Z}/2\mathbb{Z}$ , we can have the plain text in $\mathbb{Z}/2\mathbb{Z}$ using the linearity of the cipher.\\
To get the key matrix in $\mathbb{Z}/26\mathbb{Z}$ , we can use the Chinese Reminder Theorem , but we would get a complexity of $O(13^d)$. In the previous paper , it was believed that it's possible to get the key matrix in $\mathbb{Z}/26\mathbb{Z}$ without considering $\mathbb{Z}/13\mathbb{Z}$.\\
First of all , we create a hash table using long text , and search mapping between segments of reference text and plain text modulo 2\\
\#(seg in reference) = len(reference text) - n +1 , with n the segment size.\\
Indeed , if you take the following text : $thisisatest$ , with n = 5 , you get the following segment:\\
 $ thisi , hisis , isisa , sisat , isate , sates , atest $ which is 7 segments $ 11 - 5 + 1 = 7 $\\
We get the same thing for \#(str in plain) = len(plain text) - n +1 , with n the string size.\\
Then we define the good mathcings : segments are equals before and after modulo 2 , and bad matching segment are not equal but they are equal modulo 2.\\
We use Rényi entropy to get the good matching and all matching as it find the collision , with the following formula :\\
$H_{\alpha}(X) = \frac{1}{1-\alpha}log_{2}(\sum_{i=1}^{n}{Pr(X=i)^{\alpha}})$ , then when alpha has the value 2 , we just get $-log_{2}(\sum_{i=1}^{n}{Pr(X=i)^{2}})$ that gives us the probability that a segment equals another one.\\
For good matchings , we have E(\# good matchings) = (\# segments in reference) x (\#segment in plaintext) x $2^{-H_{2}(X)}$ , as the number of good matching is actually the collision between segment in plaintext and segment in reference text time the entropy of rényi where two segments are the same.\\
Then you do the same for  E(\# all matchings) , the difference is that you do it this way : E(\# good matchings) = (\# segments in reference) x (\#segment in plaintext) x $2^{-H_{2}(X mod 2)}$  . And indeed you understand that if X modulo 2 are equals the X are not always equals.\\
The previous part assumed that $H_{2}(X mod 2)$ was equals to n , but i did these maths again and it seems good.\\
Then to have an idea of the complexity , you do the ratio $\frac{E(\# good matchings)}{E(\# all matchings)}$ , you generaly found $\frac{1}{8^n}$\\
In the following parts , I did the calculation again to check if it's really correct\\
But to have a better complexity , we need to increase this ratio : E(\# all matchings) can't be changed so we can only try on E(\# good matchings) , with different assumptions and calculations.\\

\section{Experiment}
from wiki :\\

Probabilite de la 1ieme lettre de l'aphabet 0.08167\\
Probabilite de la 2ieme lettre de l'aphabet 0.01492\\
Probabilite de la 3ieme lettre de l'aphabet 0.02782\\
Probabilite de la 4ieme lettre de l'aphabet 0.04253\\
Probabilite de la 5ieme lettre de l'aphabet 0.12702\\
Probabilite de la 6ieme lettre de l'aphabet 0.02228\\
Probabilite de la 7ieme lettre de l'aphabet 0.02015\\
Probabilite de la 8ieme lettre de l'aphabet 0.06094\\
Probabilite de la 9ieme lettre de l'aphabet 0.06966\\
Probabilite de la 10ieme lettre de l'aphabet 0.00153\\
Probabilite de la 11ieme lettre de l'aphabet 0.00772\\
Probabilite de la 12ieme lettre de l'aphabet 0.04025\\
Probabilite de la 13ieme lettre de l'aphabet 0.02406\\
Probabilite de la 14ieme lettre de l'aphabet 0.06749\\
Probabilite de la 15ieme lettre de l'aphabet 0.07507\\
Probabilite de la 16ieme lettre de l'aphabet 0.01929\\
Probabilite de la 17ieme lettre de l'aphabet 9.5E-4\\
Probabilite de la 18ieme lettre de l'aphabet 0.05987\\
Probabilite de la 19ieme lettre de l'aphabet 0.06327\\
Probabilite de la 20ieme lettre de l'aphabet 0.09056\\
Probabilite de la 21ieme lettre de l'aphabet 0.02758\\
Probabilite de la 22ieme lettre de l'aphabet 0.00978\\
Probabilite de la 23ieme lettre de l'aphabet 0.02361\\
Probabilite de la 24ieme lettre de l'aphabet 0.0015\\
Probabilite de la 25ieme lettre de l'aphabet 0.01974\\
Probabilite de la 26ieme lettre de l'aphabet 7.4E-4\\
\\
\\
proba sum = 0.9999999999999999\\
proba sum squared = 0.06549717159999999\\
proba sum squared 0 = 0.56832 0.32298762240000006\\
proba sum squared 1 = 0.43167999999999995 0.18634762239999997\\
Ration of good matching and all matchings=0.1285934407027314\\
Donc 7,77644
\\
\\
\\
Another site :\\
Probabilite de la 1ieme lettre de l'aphabet 0.0808\\
Probabilite de la 2ieme lettre de l'aphabet 0.0167\\
Probabilite de la 3ieme lettre de l'aphabet 0.0318\\
Probabilite de la 4ieme lettre de l'aphabet 0.0399\\
Probabilite de la 5ieme lettre de l'aphabet 0.1256\\
Probabilite de la 6ieme lettre de l'aphabet 0.0217\\
Probabilite de la 7ieme lettre de l'aphabet 0.018\\
Probabilite de la 8ieme lettre de l'aphabet 0.0527\\
Probabilite de la 9ieme lettre de l'aphabet 0.0724\\
Probabilite de la 10ieme lettre de l'aphabet 0.0014\\
Probabilite de la 11ieme lettre de l'aphabet 0.0063\\
Probabilite de la 12ieme lettre de l'aphabet 0.0404\\
Probabilite de la 13ieme lettre de l'aphabet 0.026\\
Probabilite de la 14ieme lettre de l'aphabet 0.0738\\
Probabilite de la 15ieme lettre de l'aphabet 0.0747\\
Probabilite de la 16ieme lettre de l'aphabet 0.0191\\
Probabilite de la 17ieme lettre de l'aphabet 9.0E-4\\
Probabilite de la 18ieme lettre de l'aphabet 0.0642\\
Probabilite de la 19ieme lettre de l'aphabet 0.0659\\
Probabilite de la 20ieme lettre de l'aphabet 0.0915\\
Probabilite de la 21ieme lettre de l'aphabet 0.0279\\
Probabilite de la 22ieme lettre de l'aphabet 0.01\\
Probabilite de la 23ieme lettre de l'aphabet 0.0189\\
Probabilite de la 24ieme lettre de l'aphabet 0.0021\\
Probabilite de la 25ieme lettre de l'aphabet 0.0165\\
Probabilite de la 26ieme lettre de l'aphabet 7.0E-4\\
\\
\\
proba sum = 0.9999000000000001\\
proba sum squared = 0.06609151\\
proba sum squared 0 = 0.5657 0.32001649\\
proba sum squared 1 = 0.4342 0.18852964\\
Ration of good matching and all matchings=0.12996168115565054\\
Donc 7,69457

\section{Enhancement good matching's ratio}

This section will be to increase the ratio found which is actually $\frac{1}{8^n}$\\
To do so , I'll now consider instead of independent letters , independent blocks of letter.
With the help of a Java programm , i'm doing an heuristic search over a very very long text , with different block size.\\
With the program , we clearly see that there is no way to improve it considering that they are independent.\\

\section{Study of Faster Fourrier Transform for Algorithm 1}
lololo\\

\section{Algorithm}
You hash a reference text.\\
You take the key matrix that you get from algorithm 1 , find plain text in $\mathbb{Z}/2\mathbb{Z}$ , and create an array.\\
find the list of all matchings : find all pairs(seg,str) such that seg is a segment of plaintext modulo 2 and str $\in hash(seg)$ and save it in a list.\\
repeat\\
select d matching form list (you'll get a dxd key matrix)\\
for each of these matchings ($seg_{i}$, $str_{i}$)\\
extract $block_{i}$ from $seg_{i}$ and $str'_{i}$ from $str_{i}$,\\
then find $ciphertext_{i}$ such that $K^{-1}$ x $ciphertext_{i}$ mod 2 = $block_{i}$\\
solve $ciphertext_{i} = K * str'_{i}$ for i=1 to d\\
compute $K^{-1} * $ciphertext\\
until it makes sense\\
number of iteration is $ \frac{1}{ratio^{nd}} = 8^{nd}$
\\
\\
The following algorithm is to recover the key matrix in $\mathbb{Z}/2\mathbb{Z}$\\
Part1:\\
\\
You require Ciphertext $Y_1,Y_2,...,Y_n$\\
for all $\mu$ do
compute $S_n(\mu) = \sum_{k=1}^{n}{(-1)^{\mu.y} \times n_y}$ where $n_y=\#\{k;Y_k=y\}$\\
endfor\\
set all $\mu$ to the d values of $\mu$ with largest $S_n(\mu)=bias(\mu.Y)$\\
\\
Part2:\\
\\
for all (i,i') do\\
compute $n_{00}(i,i')=\#\{k<n:(\mu_i .Y_k,\mu_i'.Y_{k+1})=(0,0)\}$
endfor\\
set $(i_d,i_1)$ to the first pair with lowest $n_{00}$\\
\\
Part3:\\
\\
Here to be faster we store $n_y$ in a table and we do a FFT on this table to get $S_n$. With this operation the total complexity drop from $O(d^2 \times 2^d)$ to $O(d \times 2^d)$
But it seems with some other techniques we could do better
\end{document}
