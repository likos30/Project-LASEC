%%%%%%%%%%%%%%%%%%%%%%%%%%%%%%%%%%%%%%%%%
% Structured General Purpose Assignment
% LaTeX Template
%
% This template has been downloaded from:
% http://www.latextemplates.com
%
% Original author:
% Ted Pavlic (http://www.tedpavlic.com)
%
% Note:
% The \lipsum[#] commands throughout this template generate dummy text
% to fill the template out. These commands should all be removed when 
% writing assignment content.
%
%%%%%%%%%%%%%%%%%%%%%%%%%%%%%%%%%%%%%%%%%

%----------------------------------------------------------------------------------------
%	PACKAGES AND OTHER DOCUMENT CONFIGURATIONS
%----------------------------------------------------------------------------------------

\documentclass{article}

\usepackage{fancyhdr} % Required for custom headers
\usepackage{lastpage} % Required to determine the last page for the footer
\usepackage{extramarks} % Required for headers and footers
\usepackage{graphicx} % Required to insert images
\usepackage{lipsum} % Used for inserting dummy 'Lorem ipsum' text into the template
\usepackage{mathtools, bm}
\usepackage{amssymb, bm}


% Margins
\topmargin=-0.45in
\evensidemargin=0in
\oddsidemargin=0in
\textwidth=6.5in
\textheight=9.0in
\headsep=0.25in 

\linespread{1.1} % Line spacing

% Set up the header and footer
\pagestyle{fancy}
\lhead{\hmwkAuthorName} % Top left header
\chead{\hmwkClass\ (\hmwkClassInstructor\ \hmwkClassTime): \hmwkTitle} % Top center header
\rhead{\firstxmark} % Top right header
\lfoot{\lastxmark} % Bottom left footer
\cfoot{} % Bottom center footer
\rfoot{Page\ \thepage\ of\ \pageref{LastPage}} % Bottom right footer
\renewcommand\headrulewidth{0.4pt} % Size of the header rule
\renewcommand\footrulewidth{0.4pt} % Size of the footer rule

\setlength\parindent{0pt} % Removes all indentation from paragraphs

%----------------------------------------------------------------------------------------
%	DOCUMENT STRUCTURE COMMANDS
%	Skip this unless you know what you're doing
%----------------------------------------------------------------------------------------

% Header and footer for when a page split occurs within a problem environment


% Header and footer for when a page split occurs between problem environments


\setcounter{secnumdepth}{0} % Removes default section numbers
\newcounter{homeworkProblemCounter} % Creates a counter to keep track of the number of problems

\newcommand{\homeworkProblemName}{}
\newenvironment{homeworkProblem}[1][Problem \arabic{homeworkProblemCounter}]{ % Makes a new environment called homeworkProblem which takes 1 argument (custom name) but the default is "Problem #"
\stepcounter{homeworkProblemCounter} % Increase counter for number of problems
\renewcommand{\homeworkProblemName}{#1} % Assign \homeworkProblemName the name of the problem
\section{\homeworkProblemName} % Make a section in the document with the custom problem count
\enterProblemHeader{\homeworkProblemName} % Header and footer within the environment
}{
\exitProblemHeader{\homeworkProblemName} % Header and footer after the environment
}

\newcommand{\problemAnswer}[1]{ % Defines the problem answer command with the content as the only argument
\noindent\framebox[\columnwidth][c]{\begin{minipage}{0.98\columnwidth}#1\end{minipage}} % Makes the box around the problem answer and puts the content inside
}

\newcommand{\homeworkSectionName}{}
\newenvironment{homeworkSection}[1]{ % New environment for sections within homework problems, takes 1 argument - the name of the section
\renewcommand{\homeworkSectionName}{#1} % Assign \homeworkSectionName to the name of the section from the environment argument
\subsection{\homeworkSectionName} % Make a subsection with the custom name of the subsection
\enterProblemHeader{\homeworkProblemName\ [\homeworkSectionName]} % Header and footer within the environment
}{
\enterProblemHeader{\homeworkProblemName} % Header and footer after the environment
}
   
%----------------------------------------------------------------------------------------
%	NAME AND CLASS SECTION
%----------------------------------------------------------------------------------------

\newcommand{\hmwkTitle}{Bachelor Project\ \#1} % Assignment title
\newcommand{\hmwkDueDate}{Spring 2016} % Due date
\newcommand{\hmwkClass}{IN , LASEC} % Course/class
\newcommand{\hmwkClassTime}{} % Class/lecture time
\newcommand{\hmwkClassInstructor}{Pr. Serge Vaudenay} % Teacher/lecturer
\newcommand{\hmwkAuthorName}{Max Premi} % Your name

%----------------------------------------------------------------------------------------
%	TITLE PAGE
%----------------------------------------------------------------------------------------

\title{
\vspace{2in}
\textmd{\textbf{\hmwkClass:\ \hmwkTitle}}\\
\normalsize\vspace{0.1in}\small{Due\ on\ \hmwkDueDate}\\
\vspace{0.1in}\large{\textit{\hmwkClassInstructor\ \hmwkClassTime}}
\vspace{3in}
}

\author{\textbf{\hmwkAuthorName}}
\date{} % Insert date here if you want it to appear below your name

%----------------------------------------------------------------------------------------

\begin{document}

\maketitle


\newpage
\section{Abstract}
This Hill cipher is a polygraphic substitution cipher based on linear algebra , invented by 
Lester S. in 1929. Each letter is represented by a number modulo 26, it breaks the plaintext into blocks of size $d$ and then applies a matrix $d \times d $ to thiese blocks to yield ciphertext blocks. As it's a linear encryption, it can be simply broken with Know PlainText Attacks.
The author takes the previous paper about a new Ciphertext-only Attacks on Hill , and try to improve it's complexity to get a better result that $O(d13^d)$.\\
${}$\hspace{1em}The goal of this project is to actually study the algorithm to get the key matrix modulo 26 and then to improve the algorithm to get the key matrix modulo 2.\\
${}$\hspace{1em}The project report is organized as follows: Section1 presents the Hill cipher and the work done in the previous repor. In section2 , the author studies the complexity and try to improve the algorithm to get the key matrix modulo 26 . Section 3 presents the possible enhancement of the FFT of algorithm 1. Experimental results and algorithm are presented at the end.
\newpage
%----------------------------------------------------------------------------------------
%	TABLE OF CONTENTS
%----------------------------------------------------------------------------------------


%\setcounter{tocdepth}{1} % Uncomment this line if you don't want subsections listed in the ToC

\newpage
\tableofcontents
\newpage

%----------------------------------------------------------------------------------------
%	PROBLEM 1
%----------------------------------------------------------------------------------------

% To have just one problem per page, simply put a \clearpage after each problem



%----------------------------------------------------------------------------------------
%	PROBLEM 1
%----------------------------------------------------------------------------------------

% To have just one problem per page, simply put a \clearpage after each problem

\section{Introduction}
	The motivation of this project is first and foremost to improve the Linear attack on the Hill cipher , by changing the recovery of the key modulo 26 and then see the possible algorithm to improve the FFT.\\
Let's briefly recall how this attack works.\\
You get the plaintext modulo 2 , then with the aid of vectors , and bias(X)= $\varphi_{X}(\frac{2\pi}{p})$ in $\mathbb{Z}/26\mathbb{Z}$, we found correspondence between $\lambda$ and $\mu$ (the last is the same vector but for the cipher text). We actually get $ \mu = (K^T)^{-1} \times\lambda $\\
Then with this formula and the approximation of all the vector $\mu$ , we get the vectors column of the key matrix in ${Z}/2\mathbb{Z}$.\\
You just need to reorder them with the correlation , you find the last one and first one easily, and you do it recursively to find all the vectors in the correct order.\\
All this process is described by algorithm 1 at the end of the page.\\


\section{Key recovery modulo 26}
So now that we have the key matrix in $\mathbb{Z}/2\mathbb{Z}$ , we can have the plain text in $\mathbb{Z}/2\mathbb{Z}$ using the linearity of the cipher.\\
To get the key matrix in $\mathbb{Z}/26\mathbb{Z}$ , we can use the Chinese Reminder Theorem , but we would get a complexity of $O(13^d)$. In the previous paper , it was believed that it's possible to get the key matrix in $\mathbb{Z}/26\mathbb{Z}$ without considering $\mathbb{Z}/13\mathbb{Z}$.\\
First of all , we create a hash table using long text , and search mapping between segments of reference text and plain text modulo 2\\
\#(seg in reference) = len(reference text) - n +1 , with n the segment size.\\
Indeed , if you take the following text : $thisisatest$ , with n = 5 , you get the following segment:\\
 $ thisi , hisis , isisa , sisat , isate , sates , atest $ which is 7 segments $ 11 - 5 + 1 = 7 $\\
${}$\hspace{1em}We get the same thing for $\#(str in plain) = len(plain text) - n +1$ , with n the string size.\\
Then we define the good matching : segments are equals before and after modulo 2 , and bad matching segment which are not equal but equal modulo 2.\\
We use R\'enyi entropy to get the good matching and all matching as it find the collision , with the following formula :\\
$$H_{\alpha}(X) = \frac{1}{1-\alpha}log_{2}(\sum_{i=1}^{n}{Pr(X=i)^{\alpha}})$$ 
When alpha has the value 2 , we just get the following :
$$-log_{2}(\sum_{i=1}^{n}{Pr(X=i)^{2}})$$ that gives us the probability that a segment equals another one.\\
${}$\hspace{1em}For good matching , we have $E(\# good matching) = (\# segments in reference) \times (\#segment in plaintext) \times 2^{-H_{2}(X)}$ , as the number of good matching is actually the collision between segment in plaintext and segment in reference text multiplied by the entropy of r\'enyi of this segment (which represents the rate of collision for a given block X).\\
Then you do the same for  $E(\# all matching)$ , the difference is that you do it this way : $E(\# good matching) = (\# segments in reference) \times (\#segment in plaintext) \times 2^{-H_{2}(X mod 2)}$ . And indeed you understand that if 2 words modulo 2 are equals, these words are not always equals modulo 26.\\
${}$\hspace{1em}For the $E(\# all matching)$ the calculation is really simple , you must take $(\# segments in reference) \times (\#segment in plaintext) \times 2^{-H_{2}(X mod 2)}$ as we do all the possibles matching.\\
$H_{2}(X mod 2) = -log_2(\sum_{i=1}^{n}{Pr(X=i)^2})$, where $Pr(X=i)$ declined in $Pr(X=0)$ and $Pr(X=1)$\\
From diverse calculation we always get $0.5^n$ so E(\# all matching) is always equals to $(\# segments in reference) x (\#segment in plaintext) \times 0.5^n$
Then to have an idea of the complexity , you do the ratio $\frac{E(\# good matchings)}{E(\# all matchings)}$ , you generaly found $\frac{1}{8^n}$\\
In the following parts , the calculation of $E(\# all matchings)$ are done again thanks to a Java programm.
But to have a better complexity , we need to increase the ratio of good matching as $E(\# all matchings)$ can't be changed so we can only try on $E(\# good matchings)$, with different assumptions and calculations.\\


\section{Study of Faster Fourrier Transform for Algorithm 1}
With a fast fourrier Transform ( FFT ) the complexity is $O(NlogN)$ for N the input size\\

\subsection{Deteministic Sparse Fourier Approximation via Fooling Arithmetic Progressions}
If we only want to have the few significant Fourrier Coefficient , we can use this.\\
Here if we gave a threshold $\tau \in (0,1]$ and an oracle access to a function f , it outputs the $\tau$-significant Fourier Coefficient. This is called SFT and runs in $log(N) ,\frac{1}{\tau}$\\
An oracle access to a function take as input x and return the f(x) of the function f.\\
This algorithm is robust to random noise and local ( mean polynomial time)\\
It's based on partition of set by binary search , you have at the beginning 4 intervals , you test for the two first if the norm of f's Fourier Transform squared is equals to the $set_i$ oracle ouput squared\\
Meaning more explicitly : $f(J_i)^2 = \sum_{\alpha \in J_i}{|f(\alpha)|^2}$ If this pass , it will output yes , and we'll be able to continue the algorithm by replacing the J and insert the $J_i$\\
The heart of the code is actually to decide which intervals potentially contain a significant Fourier coefficient. Yes if weight on J , exceeds significant threshold $\tau$ , NO if J larger.\\
The threshold $\tau$ can be chosen , with the fact that a $\alpha$ is a $\tau -significant$ Fourier coefficient iff $|\hat{f}|^2 \geq \tau||f||^{2}_2$\\ where $\hat{f} = <f,X_{\alpha}>$ and $X_{alpha} = e^{2\pi i \alpha x/N}$

\subsection{Nearly optimal Sparse Fourier Transform}
We want here to compute the k-sparse approximation to the discrete Fourier transform of an n-dimensional signal.\\
There is to time in function of the number of input has at most k non-zero Fourier Coefficient.\\
In this case , we got $O(k.log(n))$ time , else we have $O(k.log(n).log(\frac{n}{k}))$\\
The basis is still the same, if a signal has a small number $k$ of non-zero Fourier , the output of this DFT can be represented succinctly using only $k$ coefficient.\\
What is required , is that the input size n is a power of 2.\\
This algorithm seems to restrictive and also perform the same in the worst case.\\

\subsection{Combinatorial sub linear-Time Fourier Algorithm}
You have a vector A of length $n >> k$ you identify the k largest frequencies of the transform of A , getting polynomial time $(k,log(n))$ for the algorithm.\\

\subsection{Simple and practical algorithm for sparse Fourier transform}
Here you consider a complex vector $x$ of length n
This algorithm compute the k-sparse Fourier transform in $O(\sqrt{kn}log^{3/2}n)$ , if x is sparse then you find it in exactly $O(klog^{2}n)$ , but in general estimate x is approximately $O(\sqrt{nk})$\\
So this algorithm is better if the ratio $\frac{n}{k} \in [2 \times 10^3 , 10^6]$ , but it's clearly not the best one as those before are supposed to find it in a lower complexity ($klog(n)$).\\

\section{Experiment}
\subsection{Probability of the independent English letters}
From the frequency letter given by Wikip\'edia , in english we got the following result :\\
Proba sum = 0.9999999999999999\\
Sum of probability squared = 0.06549717159999999 , which corresponds to $(\sum_{i=0}^{26}{Pr(i=y)^2})^n, y \in \{alphabet\} $\\
Sum of probability that gives 0 modulo 2 squared  = 0.32298762240000006 which corresponds to which $(\sum_{i=0}^{26}{Pr(i=0)^2})^n, i \in \{alphabet modulo 2\} $\\
Sum of probability that gives 1 modulo 2 squared = 0.18634762239999997 which corresponds to which $(\sum_{i=0}^{26}{Pr(i=1)^2})^n, i \in \{alphabet modulo 2\} $\\
Ration of good matching and all matching=$0.1285934407027314^n$\\
So $\frac{1}{7,77644^n}$
\\
\\
Another site , with some novel and book from Edgar Poe , and articles from encyclopedia :\\
\\
proba sum = 0.9999000000000001\\
sum of probability squared = 0.06609151 which corresponds to $(\sum_{i=0}^{26}{Pr(i=y)^2})^n, y \in \{alphabet\} $\\
sum of probability that gives 0 modulo 2 squared = 0.32001649 which corresponds to which $(\sum_{i=0}^{26}{Pr(i=0)^2})^n, i \in \{alphabet modulo 2\} $\\
sum of probability that gives 1 modulo 2 squared = 0.18852964 which corresponds to which $(\sum_{i=0}^{26}{Pr(i=1)^2})^n, i \in \{alphabet modulo 2\} $\\
Ration of good matching and all matching=$0.12996168115565054^n$\\
So $\frac{1}{7,69457^n}$

\subsection{Probability if you consider blocks of size d}
So calculation are done on a text of approximately $860000$ characters to see the evolution of the ratio good matching\/bad matching.\\
A program is ran to see the evolution for a block size between 1 and 25 , and give the ratio , and also the probability that a block appears.It is completely heuristic as I'm just counting the number of block that appears and do some manipulation with it. For more look at : BlockOccurence.java which is commented so everything is clear.\\
With this , the evolution of the ratio in function of the block size looks like this:\\


\section{Algorithm}
You hash a reference text.\\
You take the key matrix that you get from algorithm 1 , find plain text in $\mathbb{Z}/2\mathbb{Z}$ , and create an array.\\
find the list of all matchings : find all pairs(seg,str) such that seg is a segment of plaintext modulo 2 and str $\in hash(seg)$ and save it in a list.\\
repeat\\
select d matching form list (you'll get a dxd key matrix)\\
for each of these matchings ($seg_{i}$, $str_{i}$)\\
extract $block_{i}$ from $seg_{i}$ and $str'_{i}$ from $str_{i}$,\\
then find $ciphertext_{i}$ such that $K^{-1}$ x $ciphertext_{i}$ mod 2 = $block_{i}$\\
solve $ciphertext_{i} = K * str'_{i}$ for i=1 to d\\
compute $K^{-1} * $ciphertext\\
until it makes sense\\
number of iteration is $ \frac{1}{ratio^{nd}} = 8^{nd}$
\\
\\
The following algorithm is to recover the key matrix in $\mathbb{Z}/2\mathbb{Z}$\\
Part1:\\
\\
You require Ciphertext $Y_1,Y_2,...,Y_n$\\
for all $\mu$ do
compute $S_n(\mu) = \sum_{k=1}^{n}{(-1)^{\mu.y} \times n_y}$ where $n_y=\#\{k;Y_k=y\}$\\
endfor\\
set all $\mu$ to the d values of $\mu$ with largest $S_n(\mu)=bias(\mu.Y)$\\
\\
Part2:\\
\\
for all (i,i') do\\
compute $n_{00}(i,i')=\#\{k<n:(\mu_i .Y_k,\mu_i'.Y_{k+1})=(0,0)\}$
endfor\\
set $(i_d,i_1)$ to the first pair with lowest $n_{00}$\\
\\
Part3:\\
\\
for all $t=2$ to $d-1$ do\\
for all i $\notin \{i_1,i_2,..,i_{t-1},i_d\}$ do\\
compute $n_{00}(i,i{'})=\#\{k:(\mu^{T}_{i_{t-1}}Y_{k},\mu^{T}_{i}Y_{k})=(0,0)\}$\\
endfor\\
take i such that $n_{00}$ is minimum and set $i_t=i$\\
endfor\\
set $\mu = (\mu_{i1},\mu_{i1},...,\mu_{id})$ and $K=(\mu^-1)^T$
output K\\
\\
Here to be faster we store $n_y$ in a table and we do a FFT on this table to get $S_n$. With this operation the total complexity drop from $O(d^2 \times 2^d)$ to $O(d \times 2^d)$
But it seems with some other techniques we could do better.
\end{document}
